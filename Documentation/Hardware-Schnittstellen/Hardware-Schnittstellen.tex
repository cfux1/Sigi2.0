\documentclass[a4paper]{article}

\usepackage[utf8]{inputenc}
\usepackage[ngerman]{babel}
\usepackage[T1]{fontenc}
\usepackage{hyperref}

\title{Hardware-Schnittstellen Sigi2.0}
\date{\today}
\begin{document}
\maketitle
Der Sigi2.0 besitzt verschiedene Hardware-Komponenten, die miteinander kommunizieren:
\begin{itemize}
	\item Raspberry Pi (Raspi)
	\item Pololu Baloba 32U4 mit verschiedensten Hardware-Komponenten:
	\begin{itemize}
		\item Motoren
		\item Encoders
		\item LED's
		\item Knöpfe
		\item Buzzer
	\end{itemize}
	\item Sensoren auf dem Balboa Board
	\begin{itemize}
		\item \verb|LIS3MDL| Magnetometer
		\item \verb|LSM6DS33| Accelerometer und Gyroskop
	\end{itemize}
\end{itemize}
Der Sigi2.0 ist grob wie folgt aufgebaut:\\
Der Raspi ist für den Controller zuständig und bekommt nur Eingangs-/gibt Ausgangssignale an die restliche Hardware. Die Kommunikation vom Raspi mit der restlichen Hardware sieht dann wie folgt aus:
\begin{description}
	\item[Raspi - Sensoren] Die Sensoren auf dem Balboa Board sind direkt per I$^2$C an den Raspi angeschlossen und werden auch direkt über dieses Protokoll ausgelesen und anfangs konfiguriert.
	\item[Raspi - Balboa] Der Mikrocontroller (\verb|Atmega32U4|) auf dem Balboa Board ist auch direkt über I$^2$C an den Raspi angeschlossen, er muss jedoch noch richtig konfiguriert werden, damit der Raspi darauf zugreifen kann. Dies wurde mit dem Programm dafür realisiert, welches auf die PololuRPISlave - Library, die von Pololu bereitgestellt wird, zurückgreift und damit die Kommunikation konfiguriert.
\end{description}
Somit läuft die ganze Kommunikation des Raspi's mit den anderen Hardware Komponenten über I$^2$C ab.\\

Der \verb|Atmega32U4| nimmt dabei nur die Befehle des Raspi's auf und leitet diese über die \verb|GPIO|-Pins an die entsprechenden Komponenten weiter, bzw. nimmt die Signale der Komponenten über \verb|GPIO| auf und gibt diese dann über I$^2$C an den Raspi weiter. Er kommuniziert dabei nur mit den Hardware-Komponenten auf dem Board, die nicht direkt über I$^2$C vom Raspi angesteuert werden können, also nicht mit dem \verb|LIS3MDL| und dem \verb|LSM6DS33|. Gewisse Komponenten (Motoren \& Buzzer) werden mit Pulsweitenmodulation (PWM) angesteuert, wodurch andere Werte als nur \verb|0| und \verb|1| übermittelt werden können.
	
\end{document}