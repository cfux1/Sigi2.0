\documentclass[a4paper]{article}

\usepackage[utf8]{inputenc}
\usepackage[ngerman]{babel}
\usepackage[T1]{fontenc}
\usepackage{hyperref}

\title{Inbetriebnahme Sigi2.0}
\date{\today}
\begin{document}
\maketitle
Um den Sigi 2.0 in Betrieb zu nehmen müssen drei Hauptschritte gemacht werden:
\begin{enumerate}
	\item Betriebssystem auf den Raspberry Pi laden und Simulink konfigurieren
	\item Programm auf Pololu Balboa 32U4 flashen
	\item Simulink auf der Hardware ausführen
\end{enumerate}

\section*{Betriebssystem \& Simulink konfigurieren}
\begin{enumerate}
	\item Die \hyperref{https://ch.mathworks.com/matlabcentral/fileexchange/40313-simulink-support-package-for-raspberry-pi-hardware}{}{}{Simulink Support Package for Raspberry Pi Hardware} im Simulink installieren
	\item Mittels dem Setup Assistenten im Add-On Manager die benötigte Software auf die SD-Karte, die man dann im Raspberry Pi verwenden will, schreiben
\end{enumerate}

\section*{Pololu Balboa 32U4 flashen}
Dies geht am einfachsten mit der Arduino IDE. Diese ist \hyperref{https://www.arduino.cc/en/Main/Software}{}{}{hier} erhältlich. Sobald man diese Installiert hat geht man wie folgt vor:
\begin{enumerate}
	\item Die Datei \verb|Sigi2.ino| in einen gleichnamigen Ordner (\verb|Sigi2|) verschieben
	\item \verb|Sigi2.ino| mit Arduino öffnen
	\item In Arduino die Bibliotheksverwaltung (\textit{Sketch $\rightarrow$ Include Library $\rightarrow$ Manage Libraries}) öffnen, dort die Bibliotheken \verb|Balboa32U4| und \verb|PololuRPISlave| suchen und installieren
	\item Das Pololu Balboa 32U4 über USB an den Computer anschliessen
	\item Im Arduino unter \textit{Tools $\rightarrow$ Board:'...'} den Eintrag \verb|Board: "Arduino Leonardo"| auswählen
	\item Unter \textit{Tools $\rightarrow$ Port} den Balboa auswählen
	\item In der Hauptleiste von Arduino auf \textit{Upload} (Pfeil nach Rechts) anklicken
\end{enumerate}
Danach kann die USB-Verbindung zum Pololu Balboa wieder getrennt werden.\\

Alternativ kann man das Board auch direkt mit der \verb|avr-gcc| toolchain und \verb|AVRDUDE| programmiert werden. Dies erfordert folgende Programme, welche heruntergeladen und installiert werden müssen:
\begin{description}
	\item[Windows] \begin{itemize}
		\item \hyperref{http://winavr.sourceforge.net/}{}{}{WinAVR}
		\item \hyperref{http://www.nongnu.org/avrdude/}{}{}{AVRDUDE}
	\end{itemize}
	\item[Mac OS X] \hyperref{http://www.obdev.at/products/crosspack}{}{}{CrossPack for AVR Development}
	\item[Linux] \verb|avr-gcc|, \verb|avr-libc| und \verb|AVRDUDE|. Welche auf Ubuntu direkt aus den Paketquellen installiert werden können
\end{description}
Dann kann der Pololu Balboa 32U4 wie folgt geflasht werden:
\begin{enumerate}
	\item In der Kommandozeile zum Ordner wo man die Datei \verb|Sigi2.hex| abgelegt hat navigieren
	\item Den Balboa per USB an den Computer anschliessen
	\item Den folgenden Befehl in die Kommandozeile eingeben und darin \verb|PORT| durch den Port wo der Balboa angeschlossen ist ersetzen (Unter Windows sollte \verb| \\\\.\\USBSER000| funktionieren, falls der Balboa das einzige angeschlossene USB-Gerät ist), \textbf{noch nicht ausführen}:\\
	\verb|avrdude -p atmega32u4 -c avr109 -P PORT -D -U flash:w:Sigi2.hex|
	\item Den Balboa durch zweimaliges drücken (innerhalb 750ms) des Reset-Buttons in den Bootloader-Modus versetzen und dann direkt den zuvor eingegebenen Befehl ausführen (wartet man länger als 8s beendet der Balboa den Bootloader und er muss dann zuerst nochmals hinein versetzt werden)
\end{enumerate}

\section*{Simulink auf der Hardware ausführen}
Danach kann mittels Simulink der Code auf dem Raspberry Pi ausgeführt werden. Dazu geht man wie folgt vor.
\begin{enumerate}
	\item Simulink mit dem Modell, das man ausführen möchte, starten
	\item Auf \textit{Tools $\rightarrow$ Run on Target Hardware $\rightarrow$ Options...} gehen
	\item Unter der Anzeige \textit{Hardware Implementation} Hardware board \verb|Raspberry Pi| auswählen
	\item Darunter den Bereich \textit{Target hardware resources} ausklappen und unter der Gruppe \textit{Board Parameters} das Formular ausfüllen. Die Standardwerte sind:
	\begin{description}
		\item[Device Address] \verb|169.254.0.2|
		\item[Username] \verb|pi|
		\item[Password] \verb|raspberry|
	\end{description}
\end{enumerate}	
	
	
\end{document}